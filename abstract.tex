% (This file is included by thesis.tex; you do not latex it by itself.)

\begin{abstract}

% The text of the abstract goes here.  If you need to use a \section
% command you will need to use \section*, \subsection*, etc. so that
% you don't get any numbering.  You probably won't be using any of
% these commands in the abstract anyway.

Sparse coding principles have been widely used to model neural responses in the mammalian primary visual cortex. However, much of the computational and physiological literature has focused on understanding neuronal response characteristics in the spatial domain. The additional computational complexity required to build a space-time dependent receptive field model from neuron responses or from modeling the space-time dependent statistics of natural scenes has largely limited the amount of study that could be done in the time domain. However, it is widely believed that neurons encode information both in space and time. Here, we propose a hierarchical space-time sparse coding model for producing maximally efficient codes of video data. This is an extension on prior work in sparse coding, whereby jointly dependent space-time receptive fields will be learned from the statistical structure in videos of natural scenes. The model will require investigation of the processing role of predictive coding, lateral inhibitory competition, and multi-layer networks with recurrent top-down interactions. We will assess the network model by comparing statistical measures of space-time receptive fields against biological measures. We will also compare against neuronal population responses from multi-unit physiologically recorded data. Additionally, we will use the network as a tool for building efficient representations of video data for applications in industry, such as video compression.
\end{abstract}